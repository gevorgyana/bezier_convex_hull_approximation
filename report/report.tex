\documentclass{article}

\usepackage[utf8]{inputenc}
\usepackage[english,ukrainian]{babel}

\title{Апроксимація опуклої оболонки з використанням сплайнів Без'є}
\author{Геворгян Артем, ІПС-32}

\begin{document}

\maketitle
\pagenumbering{gobble}
\newpage
\pagenumbering{arabic}

\section{Постановка задачі}

Припустимо, існує певна множина точкок, опуклу оболонку яких потрібно
охопити певною кривою. Така задача може виникнути, наприклад, коли
недостатньо обійти точки опуклої оболонки у певному порядку (за або проти
часової стрілки) і виписати замкнену відповідну лінію. Отримана у
результаті лінія буде мати гострі кути, але хотілося б біль гладку криву.

\section{Вирішення}

Для цього використовують методи апроксимації опуклої оболонки, які
передбачають, що реальна опукла оболонка замінюється певним об'єктом, який
нагадує її властивосіті.

В дані роботі виокристана наступна властивість кривих Без'є: контрольні
точки, на яких вона побудована, утворюють опуклу оболонку для множини
точок, які складають криву. Якщо побудувати опорні точки на відрізках,
які сполучають точки справжньої опуклої оболонки вихідної множини точок,
то в результаті відбудеться шукане наближення.

Розглянемо послідовність дій алгоритму. Будуємо опуклу оболонку точок,
які отримали на вході, сортуємо її точки у певному порядку - наприклад,
проти годинникової стрілки. Проходимося по кожному симплексу і утворюємо
допоміжні точки наступним чином - ділимо поточний симплекс на три рівні
частини двома точками. Початкова точка даного симплекса, дві щойно
утворені точки, та кінцева точка поточного симплексу утворюють чотири
точки, на яких буде побудований кубічний сплайн Без'є.

\section{Деталі реалізації}

Сплайн Без'є 3-ї степені у програмі представлений явною формулою. Вона є
частковим випадком формули кривої для вищих степеней, але у реальних
графічних застосунках використовують явні формули (до того ж, частіше за
все використовують сплайни невеликих степеней - квадратичний чи кубічний).

Безпосереднє отримання точок сплайну відбувається наступним чином. Для
кожного симплексу, який обробляємо, передаємо x-координати (або
y-координати), як ваги на вхід до формули, також варіюючи параметр t від
0 до 1, для того, щоб отримати x (або y)-координату для точки сплайну,
відстань до якої її початку пропорційна t.

\end{document}
